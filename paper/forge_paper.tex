\documentclass[12pt]{article}

\usepackage[margin=1in]{geometry}
\usepackage{graphicx}
\usepackage{hyperref}
\usepackage{amsmath}
\usepackage{booktabs}
\usepackage{enumitem}

\title{FORGE: Flexible Optimization of Routes for GHG \& Energy\\
Model Structure and Data Pipeline}
\author{FORGE Developers}
\date{\today}

\begin{document}
\maketitle

\section{Overview}

Steel and aluminum production are energy-intensive industrial sectors with
significant greenhouse gas (GHG) emissions. Accurate modeling of production routes,
energy consumption, and emissions is essential for assessing decarbonization
pathways and policy impacts. However, most existing models fall in one of two
categories: detailed process-based LCA models that are complex and inflexible, or top-down sectoral models that lack process detail, usually behind proprietary data.

FORGE (Flexible Optimization of Routes for GHG \& Energy) aims to bridge this gap by
providing an open-source, process-based techno-environmental model that is both
flexible and user-friendly. It allows users to define and modify production routes,
energy sources, and emission factors through simple YAML configuration files,
enabling scenario analysis and sensitivity testing. Along with full transparency of inputs, outputs and parameters, FORGE is designed to be extensible to other industrial sectors beyond steel and aluminum. It can also be integrated with broader energy and economic models for comprehensive assessments.

FORGE is a process-based techno-environmental model for industrial production
routes, currently focused on steel and aluminum. The model is implemented as a
Python package with:
\begin{itemize}[noitemsep]
  \item a generic core engine (\texttt{forge.core.*}) that solves material and
        energy balances and computes emissions,
  \item sector descriptors and YAML datasets under \texttt{datasets/} that
        encode routes, stages, and parameters,
  \item an API layer (\texttt{forge.steel\_core\_api\_v2}) used by both CLI and
        a Streamlit UI (\texttt{forge.apps.streamlit\_app}).
\end{itemize}

This document sketches the main architecture and---most importantly---the data
pipeline from YAML input files to emissions outputs. The goal is to serve as a
starting point for a full academic paper.

At its core, FORGE implements a material balance solver that walks upstream from
final demand to determine production levels per process. Given production
levels, energy intensities, and carrier shares, it builds per-process energy
balances. This makes the model fully integrated, which allows changes in parameters (ie Blast Furnace energy intensity) to propagate both upstream and downstream. Another feature is that FORGE models process gas recovery in coke making, blast furnace and basic oxygen furncace, in line with the real world practices. A gas-routing module implements process-gas recovery, routing between direct use and electricity, and blended emission factors for gas and internal electricity. Finally, emissions are computed using energy and direct emissions, applying blended emission factors where relevant.

Aluminum implementation does not share the same intricacies of steel, and is shown here as a proof of concept of FORGE's flexibility to model other sectors. The aluminum dataset includes primary and remelt routes, downstream alloying and finishing, and direct process emissions. It does not have process gas recovery or routing, nor alternative data-sets for different scenarios (likely/optimistic/pessimistic).

\section{High-level Architecture}

\subsection{Core engine (\texttt{forge.core})}

Key modules:
\begin{itemize}[noitemsep]
  \item \textbf{\texttt{core.models}}: defines the \texttt{Process} class and
        constants such as \texttt{OUTSIDE\_MILL\_PROCS}.
  \item \textbf{\texttt{core.engine}}:
    \begin{itemize}[noitemsep]
      \item \texttt{calculate\_balance\_matrix}: material balance solver that
            walks upstream from final demand to determine production levels
            per process.
      \item \texttt{calculate\_energy\_balance}: builds per-process energy
            balances from production levels, energy intensities, and carrier
            shares.
      \item \texttt{calculate\_emissions}: computes energy and direct emissions
            by process, given energy balances, emission factors, and optional
            process-specific EFs.
    \end{itemize}
  \item \textbf{\texttt{core.gas}}:
    \begin{itemize}[noitemsep]
      \item \texttt{apply\_gas\_routing\_and\_credits}: implements process-gas
            recovery, routing between direct use and electricity, and blended
            emission factors for gas and internal electricity.
      \item \texttt{compute\_inside\_energy\_reference\_for\_share}: reference
            plant run to compute total gas and electricity for blending.
    \end{itemize}
  \item \textbf{\texttt{core.io}}:
    \begin{itemize}[noitemsep]
      \item \texttt{load\_data\_from\_yaml}: generic YAML loader with light
            normalization.
      \item \texttt{load\_recipes\_from\_yaml}: loads recipes, evaluating
            per-process expressions with access to parameters and energy tables.
    \end{itemize}
  \item \textbf{\texttt{core.routing}}:
    \begin{itemize}[noitemsep]
      \item \texttt{STAGE\_MATS}: default mapping of stage ids to materials
            (used as a fallback for legacy steel).
      \item \texttt{build\_route\_mask}: simple route masks for steel routes
            (BF--BOF, DRI--EAF, EAF-scrap, External).
      \item \texttt{enforce\_eaf\_feed}: clamps EAF recipes to a single feed
            (scrap, DRI, or pig iron) to avoid mixed-feed routes.
    \end{itemize}
  \item \textbf{\texttt{core.runner}}:
    \begin{itemize}[noitemsep]
      \item \texttt{CoreScenario}: dataclass bundling all inputs the engine
            needs (recipes, energy tables, EFs, route mask, etc.).
      \item \texttt{run\_core\_scenario}: orchestrates
            material balance $\rightarrow$ energy balance $\rightarrow$
            gas routing $\rightarrow$ emissions (and optional costs).
    \end{itemize}
\end{itemize}

\subsection{Sector descriptors (\texttt{forge.descriptor})}

Each dataset folder, e.g.\ \texttt{datasets/steel/likely} or
\texttt{datasets/aluminum/baseline}, contains a \texttt{sector.yml} that
describes:
\begin{itemize}[noitemsep]
  \item \emph{stages} (ids, materials, labels),
  \item \emph{stage menu} (what appears in the UI's ``Product'' radio),
  \item \emph{routes} (BF--BOF, DRI--EAF, EAF-scrap, etc.), including process
        enables/disables and optional feed modes,
  \item \emph{process roles} (e.g.\ gas sources),
  \item gas configuration (carriers, utility process, reference stage).
\end{itemize}

The descriptor loader (\texttt{forge.descriptor.sector\_descriptor}) turns
this YAML into a \texttt{SectorDescriptor} object used by:
\begin{itemize}[noitemsep]
  \item \texttt{scenario\_resolver} helpers (route masks, stage material
        resolution, feed mode),
  \item the Streamlit UI for stage and route selection,
  \item the API layer to build a \texttt{CoreScenario}.
\end{itemize}

\subsection{Steel API and Streamlit app}

\begin{itemize}[noitemsep]
  \item \textbf{\texttt{forge.steel\_core\_api\_v2}}:
    \begin{itemize}[noitemsep]
      \item \texttt{RouteConfig}: encapsulates route preset, stage key/role,
            demand and optional pre-selections.
      \item \texttt{ScenarioInputs}: wraps a scenario dict (YAML overrides) and
            the route config.
      \item \texttt{run\_scenario}: main entrypoint for steel; responsible for
            loading YAML, applying overrides, building the production route,
            calling the core runner, and shaping outputs.
    \end{itemize}
  \item \textbf{\texttt{forge.apps.streamlit\_app}}:
    \begin{itemize}[noitemsep]
      \item multi-sector UI (Steel / Aluminum) with sector gate,
      \item dataset and scenario selection (sidebar),
      \item downstream choices (per-stage ambiguous picks) rendered from the
            descriptor and recipe graph,
      \item model execution via \texttt{run\_scenario} and result display.
    \end{itemize}
\end{itemize}

\section{Data Pipeline}

This section focuses on how data flows from YAML to emissions. The pipeline is
essentially the same for steel and aluminum; differences are encoded in the
sector descriptors and dataset content.

\subsection{Input files per dataset}

For each dataset folder under \texttt{datasets/<sector>/<variant>/}, FORGE
expects a consistent set of core YAML files:
\begin{itemize}[noitemsep]
  \item \texttt{sector.yml}: descriptor (stages, routes, process roles, gas config).
  \item \texttt{recipes.yml}: process-level input/output coefficients.
  \item \texttt{energy\_int.yml}: energy intensity per process (MJ per run).
  \item \texttt{energy\_matrix.yml}: carrier shares per process (fractions).
  \item \texttt{energy\_content.yml}: lower heating values (MJ per unit of fuel).
  \item \texttt{emission\_factors.yml}: fuel emission factors (gCO\textsubscript{2}e/MJ).
  \item \texttt{process\_emissions.yml}: direct emissions per process
        (kgCO\textsubscript{2}e/t output).
  \item \texttt{parameters.yml}: scalar parameters used in recipe expressions
        and gas routing (e.g.\ yields, fractions).
  \item \texttt{mkt\_config.yml}: market vs endogenous production flags.
  \item optional: \texttt{process\_gases.yml} (gas meta), energy and material
        price tables, electricity intensity by country.
\end{itemize}

Scenario YAMLs (e.g.\ \texttt{scenarios/BF\_BOF\_coal.yml} or
\texttt{scenario\_aluminum.yml}) provide per-run overrides:
\begin{itemize}[noitemsep]
  \item route overrides (enable/disable specific processes),
  \item modifications to energy intensities, shares, or EFs,
  \item parameter overrides and recipe tweaks,
  \item gas-routing fractions and optional price/EF tweaks.
\end{itemize}

\subsection{From scenario to \texttt{CoreScenario}}

At a high level, the API does:
\begin{enumerate}[noitemsep]
  \item Load the sector descriptor for the dataset.
  \item Determine the route preset and stage key from the scenario name or
        scenario dict.
  \item Load base YAML tables (recipes, energy, EFs, parameters, markets).
  \item Apply scenario overrides (fuel substitutions, energy tables, parameters,
        recipes) and reload recipes to re-evaluate expressions.
  \item Build a route mask using the descriptor and route preset.
  \item Construct the production route from ambiguous picks (or defaults) using
        \texttt{\_build\_routes\_from\_picks}.
  \item Build a \texttt{CoreScenario} via \texttt{forge.scenarios.builder}
        (\texttt{build\_core\_scenario}), which:
    \begin{itemize}[noitemsep]
      \item locks route preset and EAF feed mode,
      \item carries energy tables, EFs, process EFs, gas config, and any
            process-emission whitelists,
      \item handles fallback materials and optional cost inputs.
    \end{itemize}
\end{enumerate}

\subsection{Core runner and gas routing}

Given a \texttt{CoreScenario}, \texttt{run\_core\_scenario} performs:
\begin{enumerate}[noitemsep]
  \item \textbf{Material balance}:
    \begin{itemize}[noitemsep]
      \item Build final demand dictionary for the chosen stage material
            (e.g.\ \texttt{Finished Products}, \texttt{Primary Aluminum}).
      \item Call \texttt{calculate\_balance\_matrix} with recipes, demand, and
            route mask. This returns:
        \begin{itemize}[noitemsep]
          \item a balance matrix (rows = processes + external + final demand;
                columns = materials),
          \item per-process production levels (\# runs).
        \end{itemize}
    \end{itemize}
  \item \textbf{Energy balance}:
    \begin{itemize}[noitemsep]
      \item Expand energy tables for all active processes
            (\texttt{expand\_energy\_tables\_for\_active}).
      \item Call \texttt{calculate\_energy\_balance} to obtain carrier-by-process
            energy consumption plus a TOTAL row.
    \end{itemize}
  \item \textbf{Gas routing and credits}:
    \begin{itemize}[noitemsep]
      \item Build a gas-routing scenario (gas config, route preset, demand,
            stage reference, gas fractions).
      \item Call \texttt{apply\_gas\_routing\_and\_credits}:
        \begin{itemize}[noitemsep]
          \item Identify process-gas sources and compute total recovered gas.
          \item Split recovered gas between direct use and electricity
                (utility plant) according to \texttt{direct\_use\_fraction}.
          \item Compute process-gas EF from fuel blends (excluding electricity
                and the carrier itself), and blended gas EFs from internal vs
                grid contributions.
          \item Adjust the energy balance by:
            \begin{itemize}[noitemsep]
              \item adding an internal electricity export row (Utility Plant),
              \item redirecting a fraction of gas consumption to the process-gas
                    carrier for direct use.
            \end{itemize}
          \item Return updated energy balance, updated energy EFs, and a meta
                dict with diagnostics.
        \end{itemize}
    \end{itemize}
  \item \textbf{Emissions}:
    \begin{itemize}[noitemsep]
      \item Use a robust wrapper to call \texttt{calculate\_emissions} with
            whatever combination of arguments it accepts (maintains backwards
            compatibility).
      \item \texttt{calculate\_emissions} separates market vs onsite processes,
            applies blended electricity and fuel EFs, and adds direct process
            emissions when whitelisted.
      \item A total row/column is computed when necessary; total emissions are
            returned in kg CO\textsubscript{2}e.
    \end{itemize}
\end{enumerate}

\subsection{Aluminum specifics}

Aluminum uses the same core but different descriptors and datasets:
\begin{itemize}[noitemsep]
  \item \textbf{Routes and stages} are defined in
        \texttt{datasets/aluminum/baseline/sector.yml}, with stages for
        primary, remelt, and finished aluminum.
  \item \textbf{Downstream alloying and finishing} are described via recipes
        for:
    \begin{itemize}[noitemsep]
      \item \emph{Remelt blending} and alloy series (1, 5, 6, 7),
      \item metallurgical aluminum aggregation,
      \item rolling, extrusion, ingot casting, and raw products,
      \item manufactured products and final coatings (no coating, powder
            coating, liquid painting).
    \end{itemize}
  \item \textbf{Direct process emissions} for aluminum are configured via
        \texttt{process\_emissions.yml} and a scenario-level whitelist
        (\texttt{allow\_direct\_onsite}) that flags which processes count as
        onsite direct emitters.
\end{itemize}

\section{Next Steps for the Full Paper}

This skeleton focuses on:
\begin{itemize}[noitemsep]
  \item code structure and responsibilities,
  \item file-level data pipeline,
  \item gas-routing scheme and its integration with the core.
\end{itemize}

For a complete academic paper, suggested additions:
\begin{itemize}[noitemsep]
  \item mathematical formulation of the balance solver (graph walk, uniqueness
        of producers, handling of external purchases),
  \item explicit equations for gas routing and EF blending,
  \item validation section (comparison vs external benchmarks for steel and
        aluminum),
  \item uncertainty ranges and scenario variants (likely/optimistic/pessimistic),
  \item discussion of limitations and future work (e.g.\ costs, other sectors).
\end{itemize}

\end{document}

